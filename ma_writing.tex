% Options for packages loaded elsewhere
\PassOptionsToPackage{unicode}{hyperref}
\PassOptionsToPackage{hyphens}{url}
%
\documentclass[
]{article}
\usepackage{amsmath,amssymb}
\usepackage{iftex}
\ifPDFTeX
  \usepackage[T1]{fontenc}
  \usepackage[utf8]{inputenc}
  \usepackage{textcomp} % provide euro and other symbols
\else % if luatex or xetex
  \usepackage{unicode-math} % this also loads fontspec
  \defaultfontfeatures{Scale=MatchLowercase}
  \defaultfontfeatures[\rmfamily]{Ligatures=TeX,Scale=1}
\fi
\usepackage{lmodern}
\ifPDFTeX\else
  % xetex/luatex font selection
\fi
% Use upquote if available, for straight quotes in verbatim environments
\IfFileExists{upquote.sty}{\usepackage{upquote}}{}
\IfFileExists{microtype.sty}{% use microtype if available
  \usepackage[]{microtype}
  \UseMicrotypeSet[protrusion]{basicmath} % disable protrusion for tt fonts
}{}
\makeatletter
\@ifundefined{KOMAClassName}{% if non-KOMA class
  \IfFileExists{parskip.sty}{%
    \usepackage{parskip}
  }{% else
    \setlength{\parindent}{0pt}
    \setlength{\parskip}{6pt plus 2pt minus 1pt}}
}{% if KOMA class
  \KOMAoptions{parskip=half}}
\makeatother
\usepackage{xcolor}
\usepackage[margin=1in]{geometry}
\usepackage{graphicx}
\makeatletter
\def\maxwidth{\ifdim\Gin@nat@width>\linewidth\linewidth\else\Gin@nat@width\fi}
\def\maxheight{\ifdim\Gin@nat@height>\textheight\textheight\else\Gin@nat@height\fi}
\makeatother
% Scale images if necessary, so that they will not overflow the page
% margins by default, and it is still possible to overwrite the defaults
% using explicit options in \includegraphics[width, height, ...]{}
\setkeys{Gin}{width=\maxwidth,height=\maxheight,keepaspectratio}
% Set default figure placement to htbp
\makeatletter
\def\fps@figure{htbp}
\makeatother
\setlength{\emergencystretch}{3em} % prevent overfull lines
\providecommand{\tightlist}{%
  \setlength{\itemsep}{0pt}\setlength{\parskip}{0pt}}
\setcounter{secnumdepth}{-\maxdimen} % remove section numbering
\newlength{\cslhangindent}
\setlength{\cslhangindent}{1.5em}
\newlength{\csllabelwidth}
\setlength{\csllabelwidth}{3em}
\newlength{\cslentryspacingunit} % times entry-spacing
\setlength{\cslentryspacingunit}{\parskip}
\newenvironment{CSLReferences}[2] % #1 hanging-ident, #2 entry spacing
 {% don't indent paragraphs
  \setlength{\parindent}{0pt}
  % turn on hanging indent if param 1 is 1
  \ifodd #1
  \let\oldpar\par
  \def\par{\hangindent=\cslhangindent\oldpar}
  \fi
  % set entry spacing
  \setlength{\parskip}{#2\cslentryspacingunit}
 }%
 {}
\usepackage{calc}
\newcommand{\CSLBlock}[1]{#1\hfill\break}
\newcommand{\CSLLeftMargin}[1]{\parbox[t]{\csllabelwidth}{#1}}
\newcommand{\CSLRightInline}[1]{\parbox[t]{\linewidth - \csllabelwidth}{#1}\break}
\newcommand{\CSLIndent}[1]{\hspace{\cslhangindent}#1}
\ifLuaTeX
  \usepackage{selnolig}  % disable illegal ligatures
\fi
\IfFileExists{bookmark.sty}{\usepackage{bookmark}}{\usepackage{hyperref}}
\IfFileExists{xurl.sty}{\usepackage{xurl}}{} % add URL line breaks if available
\urlstyle{same}
\hypersetup{
  pdftitle={Master's in Latin American, Caribbean and Iberian Studies},
  pdfauthor={Álvaro C.},
  hidelinks,
  pdfcreator={LaTeX via pandoc}}

\title{Master's in Latin American, Caribbean and Iberian Studies}
\author{Álvaro C.}
\date{2023-07-11}

\begin{document}
\maketitle

{
\setcounter{tocdepth}{2}
\tableofcontents
}
\hypertarget{introduction}{%
\section{Introduction}\label{introduction}}

Social movement studies have a long-standing tradition of studying
conflict, social change, and what some scholars contentious politics
(McAdam, Tarrow, and Tilly 2004). However, with the recent protests
located in the Global North have shown that

\hypertarget{methodology}{%
\section{Methodology}\label{methodology}}

\hypertarget{data}{%
\subsection{Data}\label{data}}

This article uses a quantitative methodology based on the
\href{https://www.gesis.org/en/issp/modules/issp-modules-by-topic/citizenship/2014}{Social
Survey Programme 2014 - Citizenship II (ISSP)}\footnote{By the time this
  article has been written, another wave of the same thematic survey is
  under development and expected to fully realese in 2025.} (N =
49.087). This dataset applies a similar\footnote{It is a similar
  questionary because the survey don't apply the exact same questions in
  every country. However, the questions are prepared to grasp and
  collect the same dimensions for every question applied.} questionnaire
in every country in which it is applied. The sampling procedure differs
for each country: in some countries, partly simple random samples were
applied, whereas in others, partly multistage stratified random samples.
The data was collected in two ways. The first one was through interviews
(face-to-face, Computer Assisted Personal interviews, or completed on
the telephone) or Self-administered questionnaires (Computer Assisted
self-interviews or Computer-Assisted web interviews). The analysis is
mostly applied to individuals that are 18 years or more with some
exceptions\footnote{According to the ISSP, the exceptions were countries
  such as Finland, were individuals between 15 and 75 were surveyed,
  Japan, were 16 years old and older were surveyed, South Africa, were
  the respondents were 16 years or more, and Sweden, were the
  individuals were between 17 and 79 years old.}. After processing the
dataset, the sample was reduced to 33.582 individuals nested in 34
countries.

\hypertarget{methods}{%
\subsection{Methods}\label{methods}}

The hypotheses that led this study were tested using Multivariate Linear
Regressions (MLR). As the literature says (Angrist and Pischke 2009,
2014; Chatterjee and Hadi 2006), Linear Models are appropriate when we
seek {[}\ldots.{]}. In this case, since we are testing the effect that
class structure has on the working class participation in collective
action activities, other models that provide a better understanding of
probability, such as Logistic and Logit, aren't required. To put it in
other words, since we are not measuring the \emph{probability} that the
working class has to undertake collective action activities but the
\emph{influence} that class structure itself has on

\hypertarget{analysis}{%
\section{Analysis}\label{analysis}}

1

\hypertarget{descriptives}{%
\subsection{Descriptives}\label{descriptives}}

1

\hypertarget{models}{%
\subsection{Models}\label{models}}

1

\hypertarget{discussion}{%
\section{Discussion}\label{discussion}}

\hypertarget{conclusions}{%
\section{Conclusions}\label{conclusions}}

\hypertarget{references}{%
\section*{References}\label{references}}
\addcontentsline{toc}{section}{References}

\hypertarget{refs}{}
\begin{CSLReferences}{1}{0}
\leavevmode\vadjust pre{\hypertarget{ref-angrist2009}{}}%
Angrist, Joshua D, and Jörn-Steffen Pischke. 2009. \emph{Mostly Harmless
Econometrics: An Empiricist's Companion}. Princeton university press.

\leavevmode\vadjust pre{\hypertarget{ref-angrist2014}{}}%
---------. 2014. \emph{Mastering'metrics: The Path from Cause to
Effect}. Princeton university press.

\leavevmode\vadjust pre{\hypertarget{ref-chatterjee2006}{}}%
Chatterjee, Samprit, and Ali S Hadi. 2006. \emph{Regression Analysis by
Example}. John Wiley \& Sons.

\leavevmode\vadjust pre{\hypertarget{ref-mcadam2004}{}}%
McAdam, Doug, Sidney Tarrow, and Charles Tilly. 2004. \emph{Dynamics of
Contention}. Cambridge University Press.

\end{CSLReferences}

\end{document}
